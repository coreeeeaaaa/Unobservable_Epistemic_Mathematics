\documentclass[11pt,a4paper]{article}
\usepackage[utf8]{inputenc}
\usepackage{kotex} % Korean support
\usepackage{amsmath,amssymb,amsthm}
\usepackage{enumitem}
\usepackage{booktabs}
\usepackage{geometry}
\geometry{margin=2.5cm}
\usepackage{hyperref}
\usepackage{xcolor}

% Theorem environments
\newtheorem{definition}{Definition}[section]
\newtheorem{theorem}{Theorem}[section]
\newtheorem{lemma}{Lemma}[section]
\newtheorem{axiom}{Axiom}[section]
\newtheorem{remark}{Remark}[section]
\newtheorem{example}{Example}[section]

\title{UEM Foundations v1.2: Pre-Proof Theoretical Completion \\
A Type-Theoretic and Measure-Theoretic Formalization of \\
Unobservable Epistemic Mathematics}
\author{Anonymous \\
(Type-Oriented Mathematical Formalization)}
\date{January 28, 2026}

\begin{document}

\maketitle

\begin{abstract}
We present the complete type-theoretic and measure-theoretic foundation for UEM (Unobservable Epistemic Mathematics). This v1.2 document represents the \textbf{pre-proof theoretical completion}, where all objects, operators, and axioms are defined with full mathematical rigor suitable for formal proof.

\textbf{Key innovations:}
\begin{enumerate}
\item \textbf{Observer as Typeclass}: Polymorphic observer definition with kernel equivalence relations
\item \textbf{Thickness as Outer Measure}: Carathéodory outer measure construction fully compatible with ZFC
\item \textbf{UEM Calculus}: Inductive datatype with Hangul operators as function symbols
\item \textbf{Complete Type-Operator Separation}: Explicit distinction between objects (Scalar, Vector, Tensor, Spark, Actyon, Escalade, Secare) and operators (CreateSpark, Ignite, Escalate, Collapse)
\item \textbf{Proof Sketches for 5 Core Theorems}: Type Preservation, Kernel-Margin Inequality, Projection-Exchange, Escalade Convergence, Church-Rosser
\end{enumerate}

UEM is positioned as a formal system aligned with ZFC foundations, ready for complete formal development.
\end{abstract}

\tableofcontents

\section{Introduction}

UEM (Unobservable Epistemic Mathematics) formalizes partial observability and margin-residue interactions via a geometric type theory. Previous versions (v1.0, v1.1) suffered from philosophical ambiguity and lack of formal type definitions. This v1.2 document provides:

\begin{itemize}
\item Complete type hierarchy definition (Observed, Epistemic, Meta types)
\item Measure-theoretic definition of thickness as outer measure
\item Inductive calculus definition with Hangul operators
\item Five core theorems with detailed proof sketches
\end{itemize}

\section{Type System Definition}

\subsection{Observed Types: $\mathbb{T}_{obs}$}

Standard mathematical objects compatible with existing mathematics.

\begin{definition}[Scalar]
A scalar is a value over field $\mathbb{R}$ or $\mathbb{C}$.
\[
\text{type } \text{Scalar} : \text{Type}
\]
\end{definition}

\begin{definition}[Vector]
A vector is an element of vector space $\mathbb{R}^n$.
\[
\text{type } \text{Vector}(n : \mathbb{N}) : \text{Type}
\]
\end{definition}

\begin{definition}[Tensor]
A tensor is a multilinear algebra object $V \otimes \cdots \otimes V^*$.
\[
\text{type } \text{Tensor}(k : \mathbb{N}) : \text{Type}
\]
\end{definition}

\subsection{Epistemic Types: $\mathbb{T}_{unobs}$}

UEM-specific epistemic state objects. \textbf{These are types, not operators.}

\begin{definition}[Spark]
A spark is an initial ignition point representing a topological singularity.
\[
\text{type } \text{Spark} : \text{Type}
\]
\end{definition}

\begin{definition}[Actyon]
An actyon is a state with motility/intent, an element of non-commutative Lie algebra $\mathfrak{g}$.
\[
\text{type } \text{Actyon} : \text{Type}
\]
\end{definition}

\begin{definition}[Escalade]
An escalade is a fractal recursive structure, a sequence of subsets of lattice space $\mathcal{L}$.
\[
\text{type } \text{Escalade} : \text{Type}
\]
\end{definition}

\begin{definition}[Secare]
A secare is a determined solid - an immutable state upon observation completion. 
It is a tuple containing a closed set $K$ in topological space and measure $\mu(K)$.
\[
\text{type } \text{Secare} : \text{Type}
\]
\end{definition}

\subsection{Meta Types: $\mathbb{T}_{meta}$}

Environmental variables determining operation context.

\begin{definition}[World]
The complete state space.
\[
\text{type } \text{World} : \text{Type}
\]
\end{definition}

\begin{definition}[Observer]
An entity determining observation projection function $\Pi_O$. 
This is naturally expressed as a \textbf{typeclass} for polymorphism.
\end{definition}

\begin{definition}[Margin]
The kernel space $\ker(\Pi_O)$ of observer $O$.
\[
\text{type } \text{Margin}(O : \text{Observer}) : \text{Type}
\]
\end{definition}

\section{Observer as Typeclass}

\begin{definition}[Observer Typeclass]
\[
\text{class Observer } (O : \text{Type}_w) \text{ where}
\]
\[
\text{observe } : O \to \text{ObsObject}
\]
\[
\text{kernel } : O \to O \to \text{Prop} \quad \text{(equivalence relation)}
\]
\end{definition}

\textbf{Key insight:} This typeclass definition supports polymorphism, allowing various kinds of observers within the same theoretical framework, unlike the simple definition $\text{Observer} := \{X \to \text{ObsObject}\}$.

\begin{theorem}[Observer Kernel is Equivalence]
The kernel relation $\text{kernel}(x, y) \iff \text{observe}(x) = \text{observe}(y)$ is an equivalence relation.
\end{theorem}

\section{Thickness as Outer Measure}

\subsection{Problem with Intuitive Definition}

Previous versions defined $\tau$ as "pressure" - an intuitive concept lacking mathematical manipulability.

\subsection{Measure-Theoretic Solution}

\begin{definition}[Thickness as Outer Measure]
Let $O$ be an observer with kernel equivalence relation generating equivalence classes $E = \{[x] \mid x \in O\}$.

For subset $S \subseteq E$, the thickness $\tau_O(S)$ is defined as:
\[
\tau_O(S) := \inf \left\{ \sum_{i} \mu(U_i) \mid S \subseteq \bigcup_{i} U_i, \, U_i \text{ open} \right\}
\]

This is exactly the Carathéodory outer measure construction.
\end{definition}

\begin{theorem}[Thickness is Outer Measure]
$\tau_O$ satisfies all axioms of Carathéodory outer measure:
\begin{enumerate}
\item $\tau_O(\emptyset) = 0$
\item Monotonicity: $A \subseteq B \implies \tau_O(A) \leq \tau_O(B)$
\item Countable subadditivity: $\tau_O(\bigcup_i A_i) \leq \sum_i \tau_O(A_i)$
\end{enumerate}
\end{theorem}

\textbf{Significance:} $\tau$ is no longer an intuitive concept but a fully manipulable mathematical object within ZFC measure theory.

\section{Operator Signatures}

Operators must have $f: \text{Domain} \to \text{Codomain}$ signature. Hangul syllables are symbolic representations.

\subsection{Generative and Transition Operators}

\begin{definition}[CreateSpark]
Observer generates singularity at specific point in world.
\[
\text{Op}_{\text{gen}} : W \to O \to Sp
\]
\end{definition}

\begin{definition}[Ignite]
Transform spark to directional value.
\[
\text{Op}_{\text{ig}} : Sp \to Ac
\]
\end{definition}

\begin{definition}[Escalate]
Expand directional value to fractal recursive structure.
\[
\text{Op}_{\text{esc}} : Ac \times \mathbb{N} \to Es
\]
\end{definition}

\begin{definition}[Collapse/Commit]
Project fractal structure to determined solid with thickness. \textbf{This creates Secare; Secare itself is the result type.}
\[
\text{Op}_{\text{commit}} : Es \to Se
\]
\end{definition}

\subsection{Hangul Calculus Mapping}

\begin{table}[h]
\centering
\begin{tabular}{c c c c l}
\toprule
\textbf{Traditional} & \textbf{UEM Hangul} & \textbf{Operator Name} & \textbf{Signature} & \textbf{Meaning} \\
\midrule
$\partial / \partial x$ & 사 (Sa) & Spatial Change & $Es \to V$ & Extract spatial gradient of escalade \\
$\partial / \partial t$ & 시 (Si) & Temporal Change & $Ac \to S$ & Extract temporal velocity of actyon \\
$\int$ & 마 (Ma) & Accumulation & $Sp \to Se$ & Integrate sparks to secare (mass) \\
$\nabla$ & 나 (Na) & Divergence & $Ac \to Es$ & Diffuse actyon to surrounding slots \\
$Ker(\cdot)$ & 여 (Yeo) & Margin Extract & $O \to M$ & Return unobservable region (kernel) \\
$\otimes$ & 겹 (Gyeop) & Non-commutative Overlap & $Ac \times Ac \to Ac$ & Collision/interference (order matters) \\
\bottomrule
\end{tabular}
\caption{Hangul Operator Mapping}
\end{table}

\section{UEM Calculus: Inductive Definition}

\subsection{Problem with Symbolic Interpretation}

Previous versions treated Hangul operators as symbolic aliases for existing mathematical concepts, lacking operational meaning.

\subsection{Inductive Datatype Solution}

\begin{definition}[UEM Term]
\[
\text{inductive UEMTerm } : \text{Type}_w
\]
\[
\mid \text{world } : w \to \text{UEMTerm}
\]
\[
\mid \text{spark } : \text{UEMTerm} \to \text{UEMTerm}
\]
\[
\mid \text{tensor } : \text{UEMTerm} \to \text{UEMTerm}
\]
\[
\mid \text{ga } : \text{UEMTerm} \to \text{UEMTerm} \quad \text{(ㄱㅏ)}
\]
\[
\mid \text{chi } : \text{UEMTerm} \to \text{UEMTerm} \quad \text{(ㅊㅣㄹ)}
\]
\[
\mid \text{ma } : \text{UEMTerm} \to \text{UEMTerm} \quad \text{(ㅁㅣ)}
\]
\end{definition}

Each operator has \textbf{precise input-output types}, enforced by the type discipline. For example, $\text{ga} : \text{UEMTerm} \to \text{UEMTerm}$ must receive a $\text{world}$-type term and return a $\text{spark}$-type term.

\subsection{Reduction Rules and Normalization}

\begin{definition}[Reduction Rules]
Operational meaning is given by reduction rules.
\end{definition}

\begin{theorem}[Confluence and Strong Normalization]
UEM calculus reduction rules are:
\begin{enumerate}
\item \textbf{Confluent} (Church-Rosser property): All reduction paths converge to unique normal form
\item \textbf{Strongly Normalizing}: No infinite reduction chains exist
\end{enumerate}
\end{theorem}

\textbf{Proof Sketch (Section 8.5):} Use Newman's Lemma - local confluence + strong normalization $\implies$ confluence.

\section{Core Theorems and Proof Sketches}

\subsection{Theorem 1: Type Preservation}

\begin{theorem}[Type Preservation]
For all Hangul operators and expressions, if $h : \tau$ and $h \to h'$, then $h' : \tau$.
\end{theorem}

\textbf{Proof Sketch:}
\begin{enumerate}
\item Verify type signatures for all 14 consonant and 10 vowel operators
\item Prove subject reduction: each reduction step preserves types
\item Show uniqueness of normal form types
\item Conclude by induction on reduction sequence
\end{enumerate}

\subsection{Theorem 2: Kernel-Margin Inequality}

\begin{theorem}[Kernel-Margin Inequality]
Let $K$ be positive definite kernel with eigenexpansion $K(x,y) = \sum_i \lambda_i \phi_i(x) \phi_i(y)$. Then:
\[
O_K(P, P) \leq \frac{\sum_i \lambda_i}{2} \cdot \mu(M(P))
\]
\end{theorem}

\textbf{Proof Sketch:}
\begin{enumerate}
\item Apply Mercer's theorem for spectral decomposition
\item Express overlap in spectral domain: $O_K(P,P) = \sum_i \lambda_i (\int_P \phi_i)^2$
\item Use Cauchy-Schwarz: $(\int_P \phi_i)^2 \leq \mu(P)$
\item Sum with eigenvalue weights
\item Interpret margin: $\mu(M(P)) = 2\mu(P)$ (spectral measure split)
\item Final bound follows
\end{enumerate}

\subsection{Theorem 3: Projection-Exchange}

\begin{theorem}[Projection-Exchange Theorem]
Let $V = V_{\text{keep}} \oplus V_{\text{null}}$ be orthogonal decomposition, $\Pi$ projection onto $V_{\text{null}}$. Then:
\[
|O_{\text{proj}}(\Pi, K, X, Y) - O_K(X, Y)| \leq \delta_{\text{proj}}(\Pi) + \epsilon_{\text{margin}}(\Pi)
\]
\end{theorem}

\textbf{Proof Sketch:}
\begin{enumerate}
\item Orthogonal decomposition: $X = X_{\text{keep}} + X_{\text{null}}$, $Y = Y_{\text{keep}} + Y_{\text{null}}$
\item $O_{\text{proj}} = \langle X_{\text{null}}, Y_{\text{null}} \rangle_K$
\item $O_K - O_{\text{proj}} = \langle X_{\text{keep}}, Y_{\text{keep}} \rangle_K + \text{cross-terms}$
\item Apply Cauchy-Schwarz to each term
\item Bound with $\delta_{\text{proj}} = \sup \|X_{\text{keep}}\|$, $\epsilon_{\text{margin}} = \sup \|X_{\text{null}}\|^2$
\item Final inequality follows
\end{enumerate}

\subsection{Theorem 4: Escalade Convergence}

\begin{theorem}[Escalade Convergence - Banach Fixed-Point]
Let $E$ be escalade operator on complete metric space $\Omega$. If $E$ is a contraction ($\exists k < 1, \forall x y, d(E x, E y) \leq k \cdot d(x, y)$), then:
\[
\exists! x^* \in \Omega, E(x^*) = x^*
\]
\end{theorem}

\textbf{Proof Sketch:}
\begin{enumerate}
\item Show $E$ is contraction with $k < 1$
\item Form iterative sequence $x_{n+1} = E(x_n)$
\item Prove $\{x_n\}$ is Cauchy (geometric series bound)
\item By completeness, $\exists x^*: x_n \to x^*$
\item By continuity of $E$: $x^* = E(x^*)$
\item Uniqueness: $d(x^*, y^*) \leq k \cdot d(x^*, y^*) \implies d = 0$
\end{enumerate}

\subsection{Theorem 5: Church-Rosser Property}

\begin{theorem}[Church-Rosser for Hangul Calculus]
The $\Gamma$-reduction relation is confluent: for all $h$, if $h \to^* h_1$ and $h \to^* h_2$, then $\exists h_3, h_1 \to^* h_3$ and $h_2 \to^* h_3$.
\end{theorem}

\textbf{Proof Sketch:}
\begin{enumerate}
\item Prove \textbf{local confluence}: all critical pairs join
\item Prove \textbf{strong normalization}: define measure $m(h)$ that strictly decreases
\item Apply \textbf{Newman's Lemma}: local confluence + strong normalization $\implies$ confluence
\item Conclude all reduction paths converge to unique normal form
\end{enumerate}

\section{Example: ``가이다(Ga-I-Da)'' Execution Trace}

\subsection{Setup}
\begin{itemize}
\item World: $W_{\text{real}}$
\item Observer: $O_{\text{me}}$
\item Initial state: Slot $s_{1,1,1}$ contains $\text{Void}$
\end{itemize}

\subsection{Step 1: ``가'' (Ga) - Spark Generation and Activation}

\begin{itemize}
\item \textbf{Operation:} $\text{Op}_{\text{ga}} : \text{Void} \to Ac$
\item \textbf{Decomposition:} ㄱ(GetTarget) $\to$ ㅏ(CreateSpark) $\to$ ㄱ(Ignite)
\item \textbf{Execution:} Generate Spark at $(1,1,1)$, immediately transform to Actyon
\item \textbf{Result:} $\text{obj}_1 \in \text{Actyon}$ (direction: $+x$ axis)
\end{itemize}

\subsection{Step 2: ``이'' (I) - Escalade Expansion}

\begin{itemize}
\item \textbf{Operation:} $\text{Op}_{\text{i}} : Ac \to Es$
\item \textbf{Decomposition:} ㅇ(Inherit) $\to$ ㅣ(VerticalExpand)
\item \textbf{Execution:} Expand obj$_1$ along $z$-axis and depth direction fractally
\item \textbf{Result:} $\text{obj}_2 \in \text{Escalade}$ (structure: $3 \times 3$ subgrid)
\end{itemize}

\subsection{Step 3: ``다'' (Da) - Secare Collapse}

\begin{itemize}
\item \textbf{Operation:} $\text{Op}_{\text{da}} : Es \to Se$
\item \textbf{Decomposition:} ㄷ(CloseBoundary) $\to$ ㅏ(ProjectToReal)
\item \textbf{Execution:} Close boundary of obj$_2$, fix as fact from $O_{\text{me}}$ perspective
\item \textbf{Final Result:}
\[
\text{Result} = \text{Op}_{\text{da}}(\text{Op}_{\text{i}}(\text{Op}_{\text{ga}}(\text{Void}))) \in \text{Secare}
\]
\item obj$_{\text{final}}$ is immutable solid type with permanent thickness at $(1,1,1) \in W_{\text{real}}$
\end{itemize}

\section{ZFC Conservative Extension Strategy}

\subsection{Model-Theoretic Approach}

\begin{theorem}[Model Preservation]
Construct UEM model within ZFC:
\begin{enumerate}
\item $\text{Scalar} \to \mathbb{R}$, $\text{Vector} \to \mathbb{R}^n$, $\text{Tensor} \to (\mathbb{R}^n)^{\otimes k}$
\item $\text{Spark} \to \delta_x$ (Dirac measure)
\item $\text{Actyon} \to f$ (function)
\item $\text{Escalade} \to \text{contractive map}$
\item $\text{Secare} \to \text{partitioned set}$
\item $\text{Margin} \to \text{complement of image}$
\end{enumerate}
\end{theorem}

\subsection{Proof-Theoretic Approach}

\begin{theorem}[Proof Preservation]
Define translation function $\text{UEM}_{\text{to\_ZFC}} : \text{UEMFormula} \to \text{ZFCFormula}$. Then:
\[
\text{UEM} \vdash \phi \implies \text{ZFC} \vdash \text{UEM}_{\text{to\_ZFC}}(\phi)
\]
\end{theorem}
